\documentclass{beamer}
\usetheme{metropolis}

\usepackage[greek,english]{babel}

\usepackage{listings}
\usepackage{qrcode}

\usepackage{hyperref}
\hypersetup{
    colorlinks=true,
    linkcolor=blue,
    filecolor=magenta,
    urlcolor=cyan,
}

\usepackage{minted}
\setminted{
    breaklines,
    escapeinside=||,
    mathescape,
    autogobble,
    linenos,
    numbersep=5pt
}

\title{Advanced JavaScript and Node.js}
\author{Tan E-Liang}
\institute{Hackerschool}
\date{Mar 9, 2019}

\begin{document}

\maketitle

\begin{frame}[fragile]
    Last week: JavaScript basics

    Today
    \begin{enumerate}
        \item ``Advanced'' JavaScript
        \item Make a simple web server in Node
    \end{enumerate}

    Next Week: Build a single-page web app in React
\end{frame}

\begin{frame}[fragile]{Set Up}
    \begin{enumerate}
        \item
            \href{https://nodejs.org/en/download/}{Node.js} version $\geq8$. A Node feature support matrix can be found at \href{https://node.green/}{node.green}.
            \begin{minted}{bash}
                brew install node # macOS
                sudo pacman install nodejs # Arch Linux
            \end{minted}
        \item These slides and starter code from \href{https://github.com/taneliang/hackerschool-advanced-js-node}{this GitHub repo}. Click the green ``Clone or download'' button.
        \item Some text editor
    \end{enumerate}
\end{frame}

\begin{frame}[fragile]{Recap---JavaScript Fundamentals}
    \begin{minted}{javascript}
        var neverUseVar = 12345678;
        let mutable = [1, "two", "3"];
        const immutable = "stone";

        function myFunction(arg1, arg2) {
            console.log(arg1, arg2);
        }
        setInterval(callbackFunction, 300);
    \end{minted}
\end{frame}

\begin{frame}[fragile]{Today}
    \begin{enumerate}
        \item JavaScript
            \begin{enumerate}
                \item Classes
                \item Functions as first class citizens
                \item Arrow functions (i.e. \mintinline{javascript}{(arg1, arg2) => { statements }})
                \item \mintinline{javascript}{module.exports} and \mintinline{javascript}{require} (and their ES6 equivalents (but not really) \mintinline{javascript}{export} and \mintinline{javascript}{import})
                \item \mintinline{javascript}{async}/\mintinline{javascript}{await} and Promises
            \end{enumerate}
        \item Node
            \begin{enumerate}
                \item Basics
                \item REPL
                \item NPM
                \item Express.js
            \end{enumerate}
    \end{enumerate}

    End product: Build an exam countdown API using the NUSMods API.
\end{frame}

\begin{frame}[fragile]{Why does Node.js even exist?}
    \begin{columns}
        \begin{column}{0.3\textwidth}
            \begin{center}
                \textbf{Netscape Navigator}

                Brought interactivity to web pages
            \end{center}
        \end{column}
        \begin{column}{0.3\textwidth}
            \begin{center}
                \textbf{Node.js}

                JavaScript runtime for servers
            \end{center}
        \end{column}
        \begin{column}{0.3\textwidth}
            \begin{center}
                \textbf{Electron}

                Build desktop apps with a Node+browser hybrid
            \end{center}
        \end{column}
    \end{columns}
\end{frame}

\begin{frame}[fragile]{Node.js---Key Components}
    \begin{enumerate}
        \item Read-Eval-Print-Loop (REPL)
        \item \mintinline{bash}{node} itself
        \item Node Package Manager (NPM)
    \end{enumerate}
\end{frame}

\begin{frame}[fragile]{Node REPL}
    \begin{itemize}
        \item Launch using \mintinline{bash}{node} in the terminal.
        \item Type JavaScript code at the prompt.
        \item Variables can be set and used throughout a REPL session.
        \item Exit by pressing Ctrl+D
    \end{itemize}
    \begin{minted}{javascript}
        > let name = "Chinese"
        > name
        // Output: name
        > console.log("Is it because I'm", name)
        // Output: Is it because I'm Chinese
        > name = "Malay"
        // Press the up arrow key to access your command history
        > console.log("Is it because I'm", name)
        // Output: Is it because I'm Malay
    \end{minted}
\end{frame}

\begin{frame}[fragile]{Node Scripts}
    Execute scripts by running \mintinline{bash}{node script.js}. Try running this in your terminal:
    \begin{minted}{bash}
        $ cd hackerschool-advanced-js-node/starter
        $ node 1_basics.js
    \end{minted}
\end{frame}

\begin{frame}[fragile]{Advanced JavaScript Part 1}
    We'll cover some advanced features by running through the scripts in the starter folder.
    \begin{enumerate}
        \item \lstinline{1_basics.js}
            \begin{enumerate}[(i)]
                \item Nested functions
                \item String template literals
                \item Arrow functions
            \end{enumerate}
        \item \lstinline{2_functional.js}
            \begin{enumerate}[(i)]
                \item \mintinline{javascript}{map}, \mintinline{javascript}{reduce}, \mintinline{javascript}{filter}
                \item Object/array destructuring
                \item Object/array rest/spread operator
            \end{enumerate}
        \item \lstinline{3_classes.js}
            \begin{enumerate}[(i)]
                \item Classes
            \end{enumerate}
        \item \lstinline{4.*.js}
            \begin{enumerate}[(i)]
                \item \mintinline{javascript}{module.exports}, \mintinline{javascript}{require}
            \end{enumerate}
    \end{enumerate}
\end{frame}

\begin{frame}[fragile]{Node Package Manager (NPM)}
    \href{https://www.npmjs.com/}{NPM} is a central collection of published packages/libaries/frameworks/tools.

    Run these lines in your terminal:
    \begin{minted}{bash}
        $ cd hackerschool-advanced-js-node/starter/server
        $ ls # The folder should contain only server.js
        $ npm init # and spam Enter to fill in the prompts. A package.json file should be created.
        $ npm install --save express
        $ ls # A node_modules folder should be created
    \end{minted}

    Per-project configuration file (\lstinline{package.json}) contains dependencies and other information. Per-project dependencies are stored in \lstinline{node_modules} folder.
\end{frame}

\begin{frame}[fragile]{Advanced JavaScript Part 2}
    These files are in the starter/server directory.
    \begin{enumerate}
        \item \lstinline{1_express.js}
            \begin{enumerate}[(i)]
                \item Basic Express server
                \item File server
            \end{enumerate}
        \item \lstinline{2_asyncawait.js}
            \begin{enumerate}[(i)]
                \item async/await
                \item Promises
            \end{enumerate}
        \item
            \lstinline{3_daystoexamserver.js}

            Your turn: make an exam countdown using the NUSMods~API
    \end{enumerate}
\end{frame}

\begin{frame}{Possible Extensions}
    \begin{enumerate}
        \item Add an endpoint that counts down the number of days to all exams.
    \end{enumerate}
\end{frame}

\begin{frame}
    \begin{center}
        \vspace{0.1\textheight}

        {\Huge Thank you!}

        {\Large Feedback please!}

        \qrcode[height=0.5\textheight]{http://bit.ly/hs-ajs}

        \href{http://bit.ly/hs-ajs}{http://bit.ly/hs-ajs}
    \end{center}
\end{frame}

\end{document}
